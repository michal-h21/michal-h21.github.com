\documentclass{article}
\usepackage[utf8]{inputenc}
\usepackage[T1]{fontenc}
\usepackage{hyperref}
\usepackage{minted}
\usepackage{mystyles}
\title{Using \texttt{textstyle4ht} package}
\author{Michal Hoftich}
\newcommand\ts{\texttt{textstyle4ht}\ }
\begin{document}
\maketitle

\tableofcontents
\section{Introduction}

\ts is part of
\href{https://github.com/michal-h21/helpers4ht}{helpers4ht}, helper packages
for tex4ht bundle. It's main function is to make easier custom configurations,
in particular inputing HTML tags and automatic generating corresponding CSS for
basic font attributes such as style, weight and color.

As it is best to learn with examples, here is a LaTeX package  \texttt{mystyles.sty}

\inputminted{tex}{mystyles.sty}

We can use these macros like that:

\begin{minted}{latex}
Here is some text \myhighlight{with highlights}, but \textit{also} \textbf{some
  other} formatting. 

\begin{myquote}
Some not so long quote
\end{myquote}
\end{minted}

\section{Basic configuration}
In order to make our macros configurable, we may use special file where configurable hooks are inserted. This file is named \texttt{mystyles.4ht}:

\begin{minted}{latex}
\NewConfigure{myhighlight}{2}
\pend:defI\myhighlight{\a:myhighlight}
\append:defI\myhighlight{\b:myhighlight}
\end{minted}

What it does is that it declares two  new hooks with \verb|\NewConfigure| and
then patches \verb|\myhighlight| macro. \verb|\pend:defI| will insert content
at the beginning of the macro \verb|\mystyles|, \verb|\append:defI| at the end.
Note that \texttt{:} character can be contained in macro names in \texttt{.4ht}
files. Also note that several versions of \verb|\[ap]pend:def<x>| exists,
  depending on the number of parameters of patched macro. 


Macros \verb|\a:myhighlight| and \verb|\b:myhighlight| were declared with \verb|\NewConfigure{myhighlight}{2}|. Their content can be declared with 

\begin{minted}{latex}
\Configure{myhighlight}{content for \a:myhighlight}{and content for \b:myhighlight}
\end{minted}

Now we can show usage of \ts package. We can modify the \texttt{mystyles.4ht} file slightly:

\begin{minted}{latex}
\RequirePackage{textstyle4ht}
\NewConfigure{myhighlight}{2}
\pend:defI\myhighlight{\a:myhighlight}
\append:defI\myhighlight{\b:myhighlight}
\Configure{myhighlight}{\InlineTag{myhighlight}{}}{\EndTag}
\ConfigureEnv{myquote}{\BlockTag{myquote}{}}{\EndTag}{}{}
\end{minted}


There are some new commands introduced. Normally, tags are inserted with
\verb|\HCode{<tag>}| command. \ts uses instead special commands for inserting
inline tags, block tags and their automatic closing. This meand that instead of 

\begin{minted}{latex}
\Configure{\HCode{<div class="hello">}}{\HCode{</div>}}
\end{minted}

you can use


\begin{minted}{latex}
\Configure{\BlockTag{hello}{}}{\EndTag}
\end{minted}

For block tags, \texttt{div} is the default element, so you don't have to
specify it, first parameter is class of the element, second parameter are any
HTML arguments. You can leave it empty in most cases.

If you want use different element than \texttt{div}, you can specify it as optional parameter:

\begin{minted}{latex}
\BlockTag[article]{hello}{}
\end{minted}

What is the difference between \verb|\InlineTag| and \verb|\BlockTag|? Default
element for inline tag is \texttt{span} and it should be used for shorter text
inside paragraph. Block tag should be used typically for longer texts, with
possibility of nested paragraphs. Typical use are some environments. It takes
care of correct paragraph handling see
\href{http://tex.stackexchange.com/a/66172/2891}{Control the <p> tags inserted
  by tex4ht} for some details.

\section{Styling}

\ts also provides macros for saving basic font information such as font size,
weight, style or color in the CSS file. It is little bit complicated because of
technical limitations. The font information must be grabbed at place where it
is in effect, which unfortunately isn't inside configurable hooks, as these are
inserted before and after macro code, and font changes happen inside. 

Two commands are provided: \verb|\SaveStyle| and \verb|\GrabStyle|. First one
is simpler, it saves font information for the current element. The second is
useful for saving information from commands with such as \verb|\myhighlight|

We can add to \texttt{mystyles.4ht} these lines

\begin{minted}{latex}
\append:def\myquote{\SaveStyle}
\AtBeginDocument{\GrabStyle{\InlineTag{myhighlight}{}}{\myhighlight}}
\end{minted}

They may look mysterious at first, but it is not so hard. We use
\verb|\append:def| to insert \verb|\SaveStyle| after beginning of
\texttt{mystyle} environment. It is available as \verb|\mystyle| macro. At this
moment, \texttt{<div class="myqoute">} is selected as current element and we
can use just \verb|\SaveStyle| to save the CSS.

The second line is even more mysterious. \verb|\GrabStyle| macro takes two
parameters, first produces element which should be configured CSS for, the
second is macro which is grabbed for style. It must take exactly one parameter.
The macro is printed inside a box, which is then discarded, so nothing shows in
the document. But it must be called in the moment when text can be printed, so
we must call it in \verb|\AtBeginDocument|, because \texttt{.4ht} files are
processed when LaTeX is still inside document preamble.


And this is how our example with all configurations ends:


Here is some text \myhighlight{with highlights}, but \textit{also} \textbf{some
  other} formatting. 

\begin{myquote}
Some not so long quote
\end{myquote}
\end{document}
