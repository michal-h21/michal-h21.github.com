\documentclass{article}
\usepackage{alternative4ht}
\altusepackage{fontspec}
\setmainfont{TeX Gyre Termes}
\newfontfamily\greekfont{Linux Libertine O}
\newfontfamily\russianfont{Linux Libertine O}
\newfontfamily\hindifont{Siddhanta}
\altusepackage{polyglossia}
\setmainlanguage{english}
\setotherlanguage{czech}
\setotherlanguage{greek}
\setotherlanguage{russian}
\setotherlanguage{hindi}
\usepackage{hyperref}
\usepackage{listings}
\usepackage{microtype}
\begin{document}
\title{Using \texttt{fontspec} package with tex4ht and LuaLaTeX}
\author{Michal Hoftich}
\maketitle
\tableofcontents

You can download this guide in \href{http://michal-h21.github.io/samples/helpers4ht/fontspec.pdf}{PDF}/\href{http://michal-h21.github.io/samples/helpers4ht/fontspec.tex}{TeX}.

\section{Introduction}

The \texttt{fontspec} package provides support for OpenType fonts in unicode
LaTeX formats, Xe\LaTeX\ and Lua\LaTeX. This enables full support for various
non-latin scripts, such as Indic, CJK or Arabic. Huge problem is that it is not
supported by tex4ht, convertor from \LaTeX to HTML.

The reason is that conversion is done by parsing special \texttt{DVI} file with
HTML instructions. It is done by \texttt{tex4ht} command\footnote{It is little
  bit confusing that name \textit{tex4ht} means two things: whole system
  including \LaTeX\ packages and some command line applications,
  \texttt{tex4ht} is also one of these command line applications. The command
  name will be printed in monospace font in this text to a avoid confusion}.
This command unfortunately doesn't support OpenType fonts, failing to do any
conversion at all. This needs to be fixed in the source code, but as no person
with knowledge of C language, DVI format internals and understanding of
literate sources of \texttt{tex4ht} is known to exist, it is unlikely that this
happens in foreseeable future. 

Some possible solutions which doesn't need to fix \texttt{tex4ht} exists:

\begin{enumerate}
  \item For documents which use only scripts supported by standard pdf\LaTeX,
    one can modify the document and use right arguments for \texttt{inputenc}
    and \texttt{fontenc} packages:

    \begin{lstlisting}[language=TeX]
      \ifdefined\HCode% detect tex4ht
      \usepackage[utf8]{inputenc}
      \usepackage[T1]{fontenc}
      \else
      \usepackage{fontspec}
      \setmainfont{TeX Gyre Termes}
      \fi
    \end{lstlisting}

    this method works best for European languages

  \item Use \href{https://github.com/michal-h21/lua4ht}{lua4ht}, which is an
    experimental replacement for \texttt{tex4ht} written in Lua. It doesn't
    support all tex4ht features though. See an example at
    \href{http://tex.stackexchange.com/a/253889/2891}{TeX.sx}.

  \item Use alternative version of \texttt{fontspec} and other packages
    provided by \href{https://github.com/michal-h21/helpers4ht}{helpers4ht}.
    This method will be described later in this document.


\end{enumerate}

\section{Using \texttt{alternative4ht}}

The \texttt{alternative4ht} package is included in
\href{https://github.com/michal-h21/helpers4ht}{helpers4ht}, see the linked
page for installation instructions. It  provides means to support packages
which causes tex4ht to fail. Usually these packages directly include
\texttt{PDF} instructions and it confuses \texttt{tex4ht}. In order to minimize
a need to modify a document, it enables to load alternative version of a
problematic package with tex4ht. This alternative version usually provides main
commands which may be used in a document, but the implementation if totally
different, with focus on tex4ht support.

\begin{lstlisting}[language=Tex,extendedchars=true]
\usepackage{alternative4ht}
\altusepackage{fontspec}
\setmainfont{TeX Gyre Termes}
\newfontfamily\greekfont{Linux Libertine O}
\newfontfamily\russianfont{Linux Libertine O}
\newfontfamily\hindifont{Siddhanta}
\altusepackage{polyglossia}
\setmainlanguage{english}
\setotherlanguage{czech}
\setotherlanguage{greek}
\setotherlanguage{russian}
\setotherlanguage{hindi}
....
\begin{document}

\begin{czech}
  Czech text
\end{czech}

\textgreek{Greek text}

\begin{russian}
  Russian text
\end{russian}

\begin{hindi}
  Hindi text
\end{hindi}

\end{lstlisting}

I haven't included actual text in the listings, as it would be too long and
unicode support in the \texttt{listings} package is lacking. The selected text
is first paragraph from Wikipedia pages about Prague in various languages. As
you can see, \texttt{Polyglossia} package provides two ways for including text
in other language: \verb|\text<lang>| or \verb|\begin{<lang>}|.
  The first method is better for smaller passages of text, possibly within
  paragraph, the latter is good for longer passages of text. Note that it
  always starts a new paragraph. We also need to provide a font family for each language as is shown with all these \verb|\newfontfamily\<lang>font| commands.

A result with some real world texts:


\begin{quotation}
\begin{czech}
Praha je hlavní a současně největší město České republiky a 15. největší město
Evropské unie. Leží mírně na sever od středu Čech na řece Vltavě, uvnitř
Středočeského kraje, jehož je správním centrem, ale jako samostatný kraj není
jeho součástí. Je sídlem velké části státních institucí a množství dalších
organizací a firem. Sídlí zde prezident republiky, parlament, vláda, ústřední
státní orgány a jeden ze dvou vrchních soudů. Mimoto je Praha sídlem řady
dalších úřadů, jak ústředních, tak i územních samosprávných celků; sídlí zde
též ústředí většiny politických stran a centrály téměř všech církví,
náboženských a dalších sdružení s celorepublikovou působností registrovaných v
ČR.  
\end{czech}

\textgreek{Η Πράγα (τσέχικα: Praha), είναι η πρωτεύουσα και μεγαλύτερη πόλη της
  Τσεχίας. Χτισμένη στον ποταμό Μολδάβα (Vltava), στην κεντρική Βοημία, έχει
  1,2 εκατομμύριο κατοίκους. Αποκαλείται επίσης «η χρυσή πόλη» και «μητέρα των
  πόλεων». Από το 1992, το ιστορικό κέντρο της Πράγας ανήκει στον κατάλογο
  μνημείων παγκόσμιας κληρονομιάς της UNESCO.} 

\begin{russian}
Пра́га (чеш. Praha [ˈpraɦa]) — город и столица Чехии; административный центр Среднечешского края и двух его районов — Прага-Восток и Прага-Запад. Образует самостоятельную административную единицу страны.
\end{russian}

\begin{hindi}
  प्राग युरोप के चेकोस्लोवाकिया देश की राजधानी है।
\end{hindi}
\end{quotation}

Compile the document with 

\begin{verbatim}
make4ht -ul filename 
\end{verbatim}

If your browser supports hyphenation, you can see that some texts are hyphenated. Not all languages are supported by all browsers, unfortunately.

\section{Limitations}

This method works only with Lua\LaTeX\ at the moment. It uses a node callback to
replace all characters with HTML entities, which are later translated back to
unicode values by \texttt{tex4ht}. I don't know about any easy solution for
Xe\LaTeX, other than providing huge file with \verb|\DeclareUnicodeCharacter| for all possible characters. This doesn't sound right to me, so I am open to any more realistic ideas.

I provided definitions only for most basic commands, if you find any undefined
commands in your documents, please let me know and I will try to provide reasonable definition for them\footnote{You can provide definitions as well, of course :)}.

\end{document}

