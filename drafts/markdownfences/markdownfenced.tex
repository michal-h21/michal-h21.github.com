\documentclass[]{article}

\usepackage[fencedDivs=true]{markdown}
\usepackage{tcolorbox}
\usepackage{minted}
\usepackage{hyperref}
\usepackage{linebreaker}
\usepackage{responsive}
\usepackage{fontspec}
\setmainfont{Linux Libertine O}


\ExplSyntaxOn

% the idea is to use fenced divs with classes to call custom environments with
% the class as an environment name

% define custom tcolorbox environments for warning and info
\newtcolorbox{warning}{colback=red!5!white, colframe=red!75!black,fonttitle=\bfseries, title={Warning}}
\newtcolorbox{info}{colback=blue!5!white, colframe=blue!75!black,fonttitle=\bfseries, title={info}}

% this macro will be called with the class name as argument 
% it will start the environment with that name
\def\fenceddivclass#1{
  \begin{#1}
  % we also define \endfence to end the environment
  \def\endfence{\end{#1}}
}

% set up the context for fenced divs
\def\markdownRendererFencedDivAttributeContextBegin{
  \begingroup
  % redefine the macro that will be called with the class name to use our 
  \let\markdownRendererAttributeClassName\fenceddivclass
  % if there is no class name, we just define \endfence to do nothing
  \def\endfence{}
}
% end the context by ending the environment and the group
\def\markdownRendererFencedDivAttributeContextEnd{\endfence\endgroup}
\ExplSyntaxOff

\title{Markdown Fenced Divs with Tcolorbox}
\author{Michal Hoftich}


\begin{document}
\maketitle

\section*{Introduction}

Recently, I came across a question on \href{https://tex.stackexchange.com/q/752194/2891}{TeX.SX}, asking how to create Markdown-style boxes for warnings or informational notes.  
The OP wanted to use a custom Lua filter for Pandoc, but my idea was to
experiment with this idea using the \LaTeX{} \href{https://ctan.org/pkg/markdown?lang=en}{Markdown} package, which
supports the \texttt{fencedDivs} feature.  

This feature allows you to use a class name right after three colons (e.g., \texttt{::: warning}) to trigger custom behavior.  
I thought it might be interesting to connect these fenced divs to custom \texttt{tcolorbox} environments, so that  
writing a Markdown block like

\begin{minted}{markdown}
::: warning
Some important message here.
:::
\end{minted}

would produce a nice colored box with a warning message in the final PDF, like this:

\begin{warning}
Some important message here.
\end{warning}


As the OP doesn't want to change his Pandoc based workflow, I haven't posted
the code below as an answer. But I thought it might be useful for others who
use the \texttt{Markdown} package in \LaTeX{}, so here it is on my blog.


\section*{Code Example}

The \texttt{Markdown} package enables us to write markdown directly in a \LaTeX{} document. 
It supports basic markdown syntax out of the box, but it supports extensions, which  
enable additional features. One of these extensions is \Verb|fencedDivs|, 
which allow us to create custom blocks with specific classes. 

We can require the \texttt{Markdown} package with the option \texttt{fencedDivs=true} to enable this feature:

\begin{minted}[fontsize=\small,breaklines,frame=lines]{latex}
\usepackage[fencedDivs=true]{markdown}
\end{minted}

Then, we can define custom environments using the \href{https://ctan.org/pkg/tcolorbox?lang=en}{Tcolorbox} package to style our boxes using the \Verb|newtcolorbox|:

\begin{minted}[fontsize=\small,breaklines,frame=lines]{latex}
\newtcolorbox{warning}{colback=red!5!white, colframe=red!75!black,fonttitle=\bfseries, title={Warning}}
\newtcolorbox{info}{colback=blue!5!white, colframe=blue!75!black,fonttitle=\bfseries, title={info}}
\end{minted}

These two commands declared environments \texttt{warning} and \texttt{info}. These two envrionments will print the enclosed text
in colored boxes with titles ``Warning`` and ``Info'', respectively:

\begin{minted}[fontsize=\small,breaklines,frame=lines]{latex}
\begin{info}
  this is an info box
\end{info}
\end{minted}

\begin{info}
  this is an info box
\end{info}


With these environments defined, we can now set up the \texttt{Markdown}
package to use them when it encounters fenced divs with the corresponding class
names. For each markdown feature, the \texttt{Markdown} package uses several commands
which can be redefined to customize the behavior. 

For fenced divs, we can redefine
the commands \Verb|\markdownRendererFencedDivAttributeContextBegin| and
\Verb|\markdownRendererFencedDivAttributeContextEnd| to specify what happens at the start and end of a fenced div.
The problem is that we need to know the class name specified in the fenced div (e.g., \texttt{warning} or \texttt{info}),
and this class name is passed to a macro called \Verb|\markdownRendererAttributeClassName|, which we need to redefine 
locally to start the appropriate environment:

\begin{minted}[fontsize=\small,breaklines,frame=lines]{latex}
% this macro will be called with the class name as argument 
% it will start the environment with that name
\def\fenceddivclass#1{
  \begin{#1}
  % we also define \endfence to end the environment
  \def\endfence{\end{#1}}
}

% set up the context for fenced divs
\def\markdownRendererFencedDivAttributeContextBegin{
  \begingroup
  % redefine the macro that will be called with the class name to use our 
  \let\markdownRendererAttributeClassName\fenceddivclass
  % if there is no class name, we just define \endfence to do nothing
  \def\endfence{}
}
% end the context by ending the environment and the group
\def\markdownRendererFencedDivAttributeContextEnd{\endfence\endgroup}
\end{minted}


To achieve this, we create a custom macro --- in this case,
\Verb|\fenceddivclass| --- that takes the class name as its argument and begins
the corresponding environment (for example, \Verb|\begin{warning}|).  

Inside this macro, we also define another helper command \Verb|\endfence|,
which stores the corresponding \Verb|\end{warning}| (or any other environment
name).  

By redefining \Verb|\markdownRendererAttributeClassName| to point to
\Verb|\fenceddivclass| inside the fenced div context, we make sure that each
fenced div with a class name automatically opens the correct environment when
encountered.  

At the end of the fenced div, the macro
\Verb|\markdownRendererFencedDivAttributeContextEnd| simply calls
\Verb|\endfence| and ends the current group, thus properly closing the
environment and restoring the original settings.


\section{Full Source Code}

Here is the complete \LaTeX{} code that demonstrates this approach:

\begin{minted}[fontsize=\small,breaklines,frame=lines]{latex}
\documentclass[draft]{article}

\usepackage[fencedDivs=true]{markdown}
\usepackage{tcolorbox}

\ExplSyntaxOn

% the idea is to use fenced divs with classes to call custom environments with
% the class as an environment name

% define custom tcolorbox environments for warning and info
\newtcolorbox{warning}{colback=red!5!white, colframe=red!75!black,fonttitle=\bfseries, title={Warning}}
\newtcolorbox{info}{colback=blue!5!white, colframe=blue!75!black,fonttitle=\bfseries, title={info}}

% this macro will be called with the class name as argument 
% it will start the environment with that name
\def\fenceddivclass#1{
  \begin{#1}
  % we also define \endfence to end the environment
  \def\endfence{\end{#1}}
}

% set up the context for fenced divs
\def\markdownRendererFencedDivAttributeContextBegin{
  \begingroup
  % redefine the macro that will be called with the class name to use our 
  \let\markdownRendererAttributeClassName\fenceddivclass
  % if there is no class name, we just define \endfence to do nothing
  \def\endfence{}
}
% end the context by ending the environment and the group
\def\markdownRendererFencedDivAttributeContextEnd{\endfence\endgroup}
\ExplSyntaxOff

\begin{document}

\begin{markdown}

::: warning
hello
:::

::: info
this is an info box
:::


\end{markdown}
\end{document}
\end{minted}

Compiling this document will render the Markdown text with two nicely styled boxes --
one red warning box and one blue info box:

\begin{markdown}

::: warning
hello
:::

::: info
this is an info box
:::


\end{markdown}


\end{document}
