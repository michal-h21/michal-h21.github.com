\documentclass{article}
\usepackage[utf8]{inputenc}
\usepackage[T1]{fontenc}
\usepackage[czech]{babel}
\usepackage[]{jekyll}
\usepackage[]{listings,todo}
\Abstract{%
TeX a jeho makro nadstavba LaTeX jsou tradičně považovány za nástroje, které jsou sice mocné, ale složité na používání a naučení, které produkují typograficky výborný výstup pro tisk, ale jsou nevhodné pro konverzi do formátů vhodných na web nebo pro elektronických knihy.
}
\title{Blogujeme s LaTeXem}
\begin{document}
\section*{Úvod}



\begin{lstlisting}[language={[LaTeX]TeX}, caption={Minimální dokument pro použití s jekyllem}]
\documentclass{article}
\usepackage[utf8]{inputenc}
\usepackage[czech]{babel}
\usepackage[]{jekyll}
\Abstract{%
Zde je perex dokumentu - jeho obsah se 
objevi na hlavni strance blogu a v hlavicce prispevku.

Tento prikaz musite pouzit pouze v \verb|Preamble| 
dokumentu, jinak nebude fungovat.

Jak vidite, pisi bez diakritiky - balicek \verb|listings| 
nepodporuje texty v kodovani utf8. V beznem dokumentu samozrejme diakritika v abstraktu funguje
}
\title{Blogujeme s LaTeXem}
\begin{document}
Zde nasleduje text prispevku
\end{document}
\end{lstlisting}
\lstset{showstringspaces=false}
Tento dokument můžete přeložit pomocí příkazu:
\begin{lstlisting}[language=sh, showspaces=false]
htlatex filename "xhtml, fn-in, charset=utf-8" " -cunihtf -utf8"
\end{lstlisting}


\end{document}
